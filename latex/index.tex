\href{https://travis-ci.org/Erriez/ErriezRotaryEncoderHalfStep}{\tt }

This is an optimized three speed Rotary Encoder library for Arduino which supports\+:


\begin{DoxyItemize}
\item Half step Rotary Encoder types.
\item Detect three rotation speeds.
\item Configurable rotation speed sensitivity.
\item Polling and interrupt based.
\item Single or multiple Rotary Encoders.
\item Optional Rotary button.
\item Pin state table in flash.
\end{DoxyItemize}



\subsection*{Half step / half step Rotary Encoders}

The difference between a half step or half step Rotary Encoder type is how the data signals of the two pins are generated. It depends on the mechanical construction of the notches and contacts inside the Rotary Encoder.

Please refer to the \href{https://github.com/Erriez/ErriezRotaryEncoderFullStep}{\tt Erriez\+Rotary\+Encoder\+Full\+Step} library for full step Rotary Encoders. Experiment with the full step and half step libraries which works optimal for your Rotary Encoder.

\subsection*{Hardware}

Connect the two rotary pins to the D\+I\+G\+I\+T\+AL pins of an Arduino board.

A third rotary button pin is not used in the Rotary library, but can be used in the sketch.

Tested with Arduino I\+DE v1.\+8.\+5 on hardware\+:


\begin{DoxyItemize}
\item Arduino U\+NO
\item Arduino Nano
\item Arduino Micro
\item Arduino Pro or Pro Mini
\item Arduino Mega or Mega2560
\item Arduino Leonardo
\item We\+Mos D1 R2 \& mini (E\+S\+P8266)
\end{DoxyItemize}

\subsubsection*{Interrupts}

Both rotary pins must be connected to a D\+I\+G\+I\+T\+AL pin with interrupt support, such as {\ttfamily I\+N\+T0} or {\ttfamily I\+N\+T1}. This is chip specific. Please refer to the documentation of your board or \href{https://www.arduino.cc/reference/en/language/functions/external-interrupts/attachinterrupt/}{\tt attach\+Interrupt()}.

\subsubsection*{Arduino U\+NO hardware}

The connection below can be used for polled and interrupts. An optional button pin can be connected to D\+I\+G\+I\+T\+AL pin 4.



\tabulinesep=1mm
\begin{longtabu} spread 0pt [c]{*2{|X[-1]}|}
\hline
\rowcolor{\tableheadbgcolor}\PBS\centering {\bf Rotary pin }&\PBS\centering {\bf Arduino U\+N\+O/\+N\+A\+N\+O/\+Mega2560/\+Leonardo board  }\\\cline{1-2}
\endfirsthead
\hline
\endfoot
\hline
\rowcolor{\tableheadbgcolor}\PBS\centering {\bf Rotary pin }&\PBS\centering {\bf Arduino U\+N\+O/\+N\+A\+N\+O/\+Mega2560/\+Leonardo board  }\\\cline{1-2}
\endhead
\PBS\centering 1 &\PBS\centering D\+I\+G\+I\+T\+AL pin 2 (I\+N\+T0) \\\cline{1-2}
\PBS\centering 2 &\PBS\centering D\+I\+G\+I\+T\+AL pin 3 (I\+N\+T1) \\\cline{1-2}
\PBS\centering Button (optional) &\PBS\centering D\+I\+G\+I\+T\+AL pin 4 \\\cline{1-2}
\PBS\centering G\+ND &\PBS\centering G\+ND \\\cline{1-2}
\end{longtabu}


\subsubsection*{Arduino We\+Mos D1 R2 \& mini (E\+S\+P8266) hardware}

Note that some E\+S\+P8266 pins mixes E\+S\+P8622 G\+P\+IO pins with Arduino digital pins. Connect a Rotary Encoder to the following pins which can be used with polled and interrupt examples\+:

\tabulinesep=1mm
\begin{longtabu} spread 0pt [c]{*3{|X[-1]}|}
\hline
\rowcolor{\tableheadbgcolor}\PBS\centering {\bf Rotary pin }&\PBS\centering {\bf E\+S\+P8622 pin }&\PBS\centering {\bf Text on board / We\+Mos D1 \& R2  }\\\cline{1-3}
\endfirsthead
\hline
\endfoot
\hline
\rowcolor{\tableheadbgcolor}\PBS\centering {\bf Rotary pin }&\PBS\centering {\bf E\+S\+P8622 pin }&\PBS\centering {\bf Text on board / We\+Mos D1 \& R2  }\\\cline{1-3}
\endhead
\PBS\centering 1 &\PBS\centering G\+P\+I\+O13 &\PBS\centering D7 (M\+O\+SI) \\\cline{1-3}
\PBS\centering 2 &\PBS\centering G\+P\+I\+O12 &\PBS\centering D6 (M\+I\+SO) \\\cline{1-3}
\PBS\centering Button (optional) &\PBS\centering G\+P\+I\+O14 &\PBS\centering D5 (S\+CK) \\\cline{1-3}
\PBS\centering L\+ED (Not used) &\PBS\centering G\+P\+I\+O2 &\PBS\centering D4 \\\cline{1-3}
\PBS\centering G\+ND &\PBS\centering G\+ND &\PBS\centering G\+ND \\\cline{1-3}
\end{longtabu}
{\bfseries Note\+:} An external pull-\/up resistor is required when a pin does not have an internal pull-\/up.


\begin{DoxyCode}
1 \{c++\}
2 // Connect the rotary pins to the WeMos D1 R2 board:
3 #define ROTARY\_PIN1         12
4 #define ROTARY\_PIN2         13
5 #define ROTARY\_BUTTON\_PIN   14
\end{DoxyCode}


\subsection*{Examples}

The following examples are available\+:
\begin{DoxyItemize}
\item Rotary $\vert$ Interrupt $\vert$ \href{https://github.com/Erriez/ErriezRotary/blob/master/examples/Interrupt/InterruptHalfStepBasic/InterruptHalfStepBasic.ino}{\tt Interrupt\+Half\+Step\+Basic}
\item Rotary $\vert$ Interrupt $\vert$ \href{https://github.com/Erriez/ErriezRotary/blob/master/examples/Interrupt/InterruptHalfStepButton/InterruptHalfStepButton.ino}{\tt Interrupt\+Half\+Step\+Button}
\item Rotary $\vert$ Interrupt $\vert$ \href{https://github.com/Erriez/ErriezRotary/blob/master/examples/Interrupt/InterruptHalfStepCounter/InterruptHalfStepCounter.ino}{\tt Interrupt\+Half\+Step\+Counter}
\item Rotary $\vert$ Polled $\vert$ \href{https://github.com/Erriez/ErriezRotary/blob/master/examples/Polled/PolledHalfStepBasic/PolledHalfStepBasic.ino}{\tt Polled\+Half\+Step\+Basic}
\item Rotary $\vert$ Polled $\vert$ \href{https://github.com/Erriez/ErriezRotary/blob/master/examples/Polled/PolledHalfStepButton/PolledHalfStepButton.ino}{\tt Polled\+Half\+Step\+Button}
\item Rotary $\vert$ Polled $\vert$ \href{https://github.com/Erriez/ErriezRotary/blob/master/examples/Polled/PolledHalfStepCounter/PolledHalfStepCounter.ino}{\tt Polled\+Half\+Step\+Counter}
\item Rotary $\vert$ Polled $\vert$ \href{https://github.com/Erriez/ErriezRotary/blob/master/examples/Polled/PolledHalfStepMultiple/PolledHalfStepMultiple.ino}{\tt Polled\+Half\+Step\+Multiple}
\end{DoxyItemize}

\subsection*{Documentation}


\begin{DoxyItemize}
\item \href{https://Erriez.github.io/ErriezRotaryEncoderHalfStep}{\tt Doxygen online H\+T\+ML}
\item \href{https://github.com/Erriez/ErriezRotaryEncoderHalfStep/raw/gh-pages/latex/ErriezRotaryEncoderHalfStep.pdf}{\tt Doxygen P\+DF}
\end{DoxyItemize}

\subsection*{Usage}

{\bfseries Read rotary with polling} 
\begin{DoxyCode}
1 \{c++\}
2 #include <ErriezRotaryHalfStep.h>
3 
4 
5 // Connect rotary pins to the DIGITAL pins of the Arduino board
6 #define ROTARY\_PIN1   2
7 #define ROTARY\_PIN2   3
8 
9 // Enable ONE of the three constructors below with different number of arguments:
10 
11 // Initialize half step rotary encoder, default pull-up enabled, default 
12 // sensitive=100
13 RotaryHalfStep rotary(ROTARY\_PIN1, ROTARY\_PIN2);
14 
15 // Or initialize half step rotary encoder, pull-up disabled, default sensitive=100
16 // RotaryHalfStep rotary(ROTARY\_PIN1, ROTARY\_PIN2, false);
17 
18 // Or initialize half step rotary encoder, pull-up enabled, sensitive 1..255
19 // A higher value is more sensitive
20 // RotaryHalfStep rotary(ROTARY\_PIN1, ROTARY\_PIN2, true, 150);
21 
22 void loop()
23 \{
24   int rotaryState = rotary.read();
25 
26   // rotaryState = -3: Counter clockwise turn, multiple notches fast
27   // rotaryState = -2: Counter clockwise turn, multiple notches
28   // rotaryState = -1: Counter clockwise turn, single notch
29   // rotaryState = 0:  No change
30   // rotaryState = 1:  Clockwise turn, single notch
31   // rotaryState = 2:  Clockwise turn, multiple notches
32   // rotaryState = 3:  Clockwise turn, multiple notches fast
33 \}
\end{DoxyCode}


{\bfseries Read rotary with interrupts}


\begin{DoxyCode}
1 \{c++\}
2 #include <ErriezRotaryHalfStep.h>
3 
4 // Connect rotary pins to Arduino DIGITAL pins with interrupt support:
5 //
6 // +-----------------------------------+--------------------------+
7 // |              Board                |  DIGITAL interrupt pins  |
8 // +-----------------------------------+--------------------------+
9 // | Uno, Nano, Mini, other 328-based  |  2, 3                    |
10 // | Mega, Mega2560, MegaADK           |  2, 3, 18, 19, 20, 21    |
11 // | Micro, Leonardo, other 32u4-based |  0, 1, 2, 3, 7           |
12 // +-----------------------------------+--------------------------+
13 //
14 #define ROTARY\_PIN1   2
15 #define ROTARY\_PIN2   3
16 
17 // Enable ONE of the three constructors below with different number of arguments:
18 
19 // Initialize half step rotary encoder, default pull-up enabled, default 
20 // sensitive=100
21 RotaryHalfStep rotary(ROTARY\_PIN1, ROTARY\_PIN2);
22 
23 // Or initialize half step rotary encoder, pull-up disabled, default sensitive=100
24 // RotaryHalfStep rotary(ROTARY\_PIN1, ROTARY\_PIN2, false);
25 
26 // Or initialize half step rotary encoder, pull-up enabled, sensitive 1..255
27 // A higher value is more sensitive
28 // RotaryHalfStep rotary(ROTARY\_PIN1, ROTARY\_PIN2, true, 150);
29 
30 void setup()
31 \{
32   // Initialize pin change interrupt on both rotary encoder pins
33   attachInterrupt(digitalPinToInterrupt(ROTARY\_PIN1), rotaryInterrupt, CHANGE);
34   attachInterrupt(digitalPinToInterrupt(ROTARY\_PIN2), rotaryInterrupt, CHANGE);
35 \}
36 
37 void rotaryInterrupt()
38 \{
39   int rotaryState = rotary.read();
40 
41   // rotaryState = -3: Counter clockwise turn, multiple notches fast
42   // rotaryState = -2: Counter clockwise turn, multiple notches
43   // rotaryState = -1: Counter clockwise turn, single notch
44   // rotaryState = 0:  No change
45   // rotaryState = 1:  Clockwise turn, single notch
46   // rotaryState = 2:  Clockwise turn, multiple notches
47   // rotaryState = 3:  Clockwise turn, multiple notches fast
48 \}
\end{DoxyCode}


\subsection*{Library dependencies}


\begin{DoxyItemize}
\item None.
\end{DoxyItemize}

\subsection*{Library installation}

Please refer to the \href{https://github.com/Erriez/ErriezArduinoLibrariesAndSketches/wiki}{\tt Wiki} page.

\subsection*{Other Arduino Libraries and Sketches from Erriez}


\begin{DoxyItemize}
\item \href{https://github.com/Erriez/ErriezArduinoLibrariesAndSketches}{\tt Erriez Libraries and Sketches} 
\end{DoxyItemize}